\documentclass[11pt]{article}
\usepackage[left=1.0in,top=1.0in,
bottom=1.0in,right=1.0in]{geometry}

\usepackage[utf8]{inputenc}
\begin{document}
\title{Cognitive Sciecne\\ Short Notes\\ on\\ Hermann von Helmholtz}
\author{Sankarsan Seal CS1511}
\date{1st September, 2016}
\maketitle

\section{Hermann von Helmhotlz (August 31, 1821  – September 8, 1894)}
\paragraph{Hermann von Helmholtz is know for his works on different fields which include electrodynamic, thermodynamic, physics and most importantly in arena of Philosophy and psychology. As We are student of cognitive science we are focusing more on his empirical work of psychology and philosophy. His every work opposed pre-existing idea of vital force and innate response of external sensation.}

\subsection{Personal Details}
Hermann Ludwig Ferdinand von Helmholtz, was physician and physicist, was born in German on 31st August,1821 \footnote{Belated Birthday Wish!!!}. He was keen to pursue study of natural science but he had his study of medicine for financial support. But when he got opportunity to do research after serving to army as physician, he concentrated in natural sciences like mechanics, physics.

\subsubsection{Conservation of Energy}
Although he was not the first person who conceived the idea of conservation of energy but he is the one who explained the energy very clearly and methodically.

He observed the relation between heat generation in muscles and  usage of food. He deduced that no energy is created indefinitely or destroyed. Energy type can only transformed to another energy type. As collision of objects, explosion, contraction of muscles use some kind of energy and convert unused energy as heat. Total energy equates to  sum of utilized energy and wasted energy. \\textit{It is eventually known as First Law of Thermodynamics.}

His observations are published as his own book (\textit{ Über die Erhaltung der Kraft (On the Conservation of Force)}) in 1847. 

\subsubsection{Optics Instruments}
In 1851 greatest invention of von Helmholtz is devising ophthalmoscope and ophthalmometer to examine the retina and observing curvature of inner wall of eye. 

Due to his invention, we find new way of thinking the study perception and sensation instead of prior conception of vital force as sensation, which was basis of his student Wilhelm Wundt.

\subsubsection{Study of Physiological Optics and Dioptrics}
Using above mentioned apparatus, he identified eyes have capability to recognise three fundamental colors ( Red, Green and  Blue) with Thomas Young.

\subsubsection{Ear and Study of tone}
As expert pianist, he was very eager to understand how we distinguish different tones and pitches of sound produced by different playing instrument. He even try to find why we feel different when same pitch sound is produced by different instrument too.

He studied inner portion of ear and comprehended that some particular bones and spiral-snail like internal organ called cochlea resonate particular frequency sounds and that helps us to discern different tone, pitch and quality of sounds.



\subsubsection{Emperical Measurement of Nerve Signal}
Using the frog's leg-muscle contraction experiment(in 1849), he first measure the speed of propagation speed of nerve signal from source through nervous system. Electricity was generated from galvanometer which is also another contribution of von Helmholtz.
Initially it was believed, every internal signal was governed by vital functions and it is beyond the purview of experimental measurement. But he calculated with the instrument called myograph.

\subsubsection{Electromagnetism}
During the period 1971, he was intrigued with electromachanism studies and he came up with well know \textit{Helmholtz equation}. His work on electromagnetism which paved the way of other talented scientists like Heinrich Rudolf Hertz and James Clerk Maxwell. Hertz was student of Helmholtz.

\subsubsection{Contribution in Mathematics}
Although he never get any teaching from expert of mathematics, he also contributed in mathematics too. He self-taught about mathematics going through books of Laplace, Bernoulli and Biot.





\end{document}