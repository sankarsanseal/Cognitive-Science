\documentclass[11pt]{article}
\usepackage[left=1.0in,top=1.0in,
bottom=1.0in,right=1.0in]{geometry}
\begin{document}
\title{Cognitive Sciecne\\ Short Notes\\ on\\ Hermann von Helmholtz}
\author{Sankarsan Seal CS1511}
\date{1st September, 2016}
\maketitle

\section{Hermann von Helmhotlz}
\paragraph{Hermann von Helmholtz is know for his works on different fields which include electrodynamic, thermodynamic, physics and most inportantly in arena of Philosophy and psychology. As We are student of cognitive science we are focusing more on his empirical work of pshychology and philosophy.}

\subsection{Personal Details}
He was physician and physicist, was born in German on 31st August,1821. He was keen to pursue study of natural science but he had his study of medicine for financial support.

\subsubsection{Optics Instrucments}
Greatest invention of von Helmholtz is devising ophthalmoscope and ophthalmometer to examine the retina and observing curvature of inner wall of eye.

Due to his invention, we find new way of thinking the study perception and sensation instead of prior conception of vital force as sensation.

\subsubsection{Ear and Study of tone}
As expert pianist, he was very eager to understand how we distinguish different tone and pitch of sound produced by different playing instrument. He even try to find why we feel different when same pitch sound is produced by different instrument too.

He studied inner portion of ear and comprehend that some particular bones and spiral-snail like internal organ called cochlea resonate particular frequency sounds and that helps us to discern different tone, pitch and quality of sounds.

 




\end{document}